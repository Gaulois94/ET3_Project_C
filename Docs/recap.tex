\section{Qui fait quoi ?}

\subsection{Stacy GROMAT}
\paragraph{} Stacy GROMAT a créé tous les contexts ainsi que le fonctionnement du joueur. InGame, Start et Options sont donc ses travaux principaux. Elle a de plus codé les classes List et ResourcesManager, qui sont tout simplement des implémentations de std$::$list et std$::$map (à leur version la plus basique) disponible en C++. 

\subsection{Mickaël SERENO}
\paragraph{} L'éditeur de carte étant le sien, Mickaël SERENO a décompilé la carte, créé les différentes classes Ennemy / Tiles et les classes Drawables / Widgets.

\subsection{Ensemble}
\paragraph{} Nous ne pouvions pas partir chacun de notre coté sans avoir fait ensemble une base solide. Nous avons donc décidé ensemble de comment le projet sera construit, comment nous gèrerons les évènements, comment les classes intéragiront entre elles, etc.
\paragraph{} La plupart des .h (malheureusement pas tous) ont été commentés avec la syntaxe doxygen. Pourquoi ? Pour avoir une documentation écrite par le programme doxygen afin de se servir des fonctionnalités créées. Ceci est surtout important pour les classes telles que celles dans le dossier Drawables et Widgets qui font parties du moteur graphique. Nous n'avons pas recommenté les fonctions qui sont redéfinies par les classes filles, comme les fonctions howActive ou draw. Elles l'ont été dans leur classe mère.

\subsection{Provenances des ressources}
\paragraph{} Nous n'avons malheureusement pas pu créer nos propres fichiers images, fonts et sons. Tous les fichiers autres que xml présents dans le dossier Resources viennent donc de internet. Les fichiers images viennent tous de \url{http://www.mariouniverse.com/} 
