\section{Difficultés / Solutions}

\subsection{Difficultés}
\paragraph{} La principale difficulté d'un jeu tel que Mario Bros est de le rendre suffisament modulable pour avoir des perspectives d'améliorations facilement implémentables. De plus, le jeu pouvant vraiment être complet, le format de carte est un choix important du projet.
\paragraph{} La SDL nous étant imposées, il fallait la comprendre et surtout l'utiliser le plus simplement possible (dans le sens que le code devait rester logique) pour avoir un projet le plus modulable.

\subsection{Solutions}
\subsubsection{Orienté Objet}

\paragraph{} Pour la modularité du projet, nous avons opté pour une approche objet du problème. Bien que le C n'est pas un langage orienté objet, nous avons pu utiliser les concepts de bases d'un code orienté objet, à savoir :
\begin{itemize}
	\item Le principe d'héritage
	\item Le principe de polymorphisme
	\item Les méthodes et attributs
\end{itemize}

\paragraph{} Pour l'héritage, nous avons tout simplement profité des caractéristiques des structures en langage C, à savoir que l'on peut caster et avoir accès aux N premiers bits d'une structure comme si celle-ci était un autre type de données.
\paragraph{} Pour le polymorphisme, les pointeurs sur fonctions étaient ce qui semblaient le plus approprié. Il suffit juste de changer ce pointeur (qui fait partie des attributs de la structure) vers la bonne méthode et tout fonctionnera.

\subsubsection{Moteur de jeu}

\paragraph{} Avec le concept d'objet utilisable, nous avons pu développer un petit mais fonctionnel moteur de jeu. Les animations, textes, sprites, Widgets sont utilisables dans l'ensemble, et ne dépendent pas du jeu lui même.
\paragraph{} Nous avons de plus utilisé le concept de Context afin d'avoir des parties du code bien définies. Un Context est tout simplement un morceau du jeu qui se suffit à lui même. Le menu Start, le menu d'Options et le jeu en lui même (InGame) sont des Contexts qui ne dépendent de personne pour subsister. Les variables qu'ils échangent sont enregistrées dans globalVar.c / globalVar.h
\paragraph{} Pour décompiler la carte, nous avons utilisé expat \footnote{\url{http://expat.sourceforge.net/}} qui est un parser évènementiel. À chaque fois qu'il rencontre une balise ouvrante (fermante), il appelera la fonction correspondante définie par XML\_SetElementHandler en donnant en paramètre une donnée qu'on aura choisi (ici ce sera notre structure Map), les attributs de la balise et son nom. Le format de la carte est donné en commentaire dans include/Map.h.
\paragraph{} Nous en avons profité pour créer un parser CSV basique. En effet, les rangées de blocs sont écrites en CSV, et il nous fallait donc écrire un parser.
\paragraph{} Chaque tile a un type écrit dans le fichier xml. La carte récupère ce type ce qui permet de créer le bon object au bon moment. Chaque object n'a donc pas un type standard, mais est propre à ce qu'il est. Leur utilisation est donc grandement simplifiée, et rajouter des objects est assez simple en soit.

\subsubsection{Les évènements}

\paragraph{} Pour les évènements, nous n'avons pas voulu qu'une seule entité gère tous les évènements. Nous donnons donc successivement les évènements aux objets qui peuvent potentiellement récupérer à la volée cet évènement et laisser cette entité gérer son évènement toute seule. Comme les évènements au clavier ne sont pas continus, nous utilisons des variables en plus pour connaître le dernier état d'un évènement (par exemple stillDown dans Player.h qui permet de savoir si la touche est toujours enfoncée)

\subsubsection{IAs}

\paragraph{} Chaque ennemi (en théorie, dans ce code il n'y en a qu'un seul) est lié à une fonction IA. En effet la difficulté d'une IA, en plus d'avoir un code complexe, est la manière d'accéder aux informations distantes (le Contexte InGame, le Player, etc.). Nous avons donc utilisé des pointeurs sur fonction qui serviront de fonction IA.
