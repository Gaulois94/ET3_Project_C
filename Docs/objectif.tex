\section{Objectifs}

\paragraph{} Les objectifs du projet furent principalement d'implémenter les concepts de bases d'un jeu Mario Bros, c'est à dire le fait de traverser une carte créée avec des blocs \footnote{Tiles dans le code} et des ennemis afin de passer au niveau suivant. En ce sens, nous avons réussi à reproduire ce concept à son stade le plus basique. En effet nous avons pu recréer les systèmes des pièces, des sols, des points de départs (qui n'est certe pas affiché mais existe dans notre format de carte) et des points d'arrivés (le château). De plus, nous avons aussi créé les ennemis les plus basiques appelés Goombas dans le code.

\paragraph{} Il était aussi prévu de faire une gestion des records, de plusieurs cartes (nous voulions en proposer 2) et même d'une mode deux joueurs.

\subsection{Ceux réussis}

\paragraph{} Nous avons donc réussi de nombreux mécanismes dans ce projet, notamment tout ce qui tourne autour de la gestion de la carte et des intéractions avec elles. Le personnages, les ennemies ainsi que blocs / pièces fonctionnent. Un point de départ et d'arrivé ont aussi été définis.
\paragraph{} De la musique évènementiel ont aussi été créé (les sauts et les pièces pour le moment, mais en rajouter est très simple). Une musique de fond a aussi té rajouté (celle du jeu original, Nintendo Copyright).
\paragraph{} Le menu Option a aussi été créé, où nous pouvons modifier les touches du personnage, et désactiver le son. D'autres options pourront aussi y apparaître ultérieurement si besoin.

\subsection{Ceux pas fait}
\paragraph{} Nous avons ici géré qu'une seule carte. En gérer une autre est simple, mais par manque de temps, nous avons du faire des compromis. Nous n'avons pas plus non plus géré les records comme prévu au début. Nous avons préféré avoir un jeu fonctionnel au lieu d'avoir des choses en plus et avoir rien du tout de fonctionnelle pour l'utilisateur. Enfin, le mode deux joueurs n'est pas présent, comme prévu initialement. On c'est rendu compte que ceci ne rapporterai pas beaucoup dans le jeu, et donc on a décider de ne pas le faire (quelques jours après que le cahier des charges ait été envoyé, nous avions décidé de ne pas le faire).
